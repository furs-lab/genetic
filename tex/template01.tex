\documentclass[russian,a4paper,12pt]{article}
\usepackage[utf8]{inputenc}
\usepackage[russian]{babel}
\usepackage{tabularx}
\usepackage{longtable}
\usepackage{caption}
\usepackage[dvipsnames,table]{xcolor}
\usepackage{graphicx}

\definecolor{light-gray}{HTML}{E5E4E2}
\definecolor{light-cyan}{HTML}{E0FFFF}

%\usepackage{graphicx,amssymb,subcaption,xcolor} 
%\bibliographystyle{unsrtnat}
%\usepackage[numbers,sort&compress]{natbib}
%\usepackage{amsmath}
%\graphicspath{{pictures/}}
%\DeclareGraphicsExtensions{.pdf,.png,.jpg,.eps}


\begin{document}

\begin{tabularx}{\textwidth}{ X r }
	\hline
	ФИО: & \VAR{name} \\ 
	Дата рождения: & \VAR{birthday}\\  
	Пол: & \VAR{sex} \\
	Номер анализа: & \VAR{analysis_number} \\
	Дата выполнения исследования: & \VAR{analysis_date} \\
	
	\hline
	   
\end{tabularx}

\vspace{15mm}
\begin{center}
	\large{МОЛЕКУЛЯРНО-ГЕНЕТИЧЕСКОЕ ИССЛЕДОВАНИЕ}
	\\
	\vspace{5mm}
	
	\large{Оценка риска гипертонической болезни и фармакогенетика антигипертензивных средств}
	\\
	\vspace{5mm}
	
\end{center}

\begin{center}
	%\setlength\LTleft{0pt}
	%\setlength\LTright{0pt}
	\rowcolors{1}{light-cyan}{light-gray}
	\begin{longtable}{|l|l|l|l|}%{@{\extracolsep{\fill}}|l|l|l|l|@{}}%{|l|l|l|l|}
		\hiderowcolors
		\caption*{\textsl{Результаты генетического тестирования}} \\
		
		\rowcolor{yellow}\hline
		\hline \multicolumn{1}{|c|}{\textbf{Ген}} & \multicolumn{1}{c|}{\textbf{RS}} & \multicolumn{1}{c|}{\textbf{Полиморфизм}} & \multicolumn{1}{c|}{\textbf{Генотип}}
		\\ \hline 
		\endfirsthead
		\hiderowcolors
		
		\multicolumn{4}{c}%
		{} \\
		\rowcolor{yellow}\hline
		\hline \multicolumn{1}{|c|}{\textbf{Ген}} & \multicolumn{1}{c|}{\textbf{RS}} & \multicolumn{1}{c|}{\textbf{Полиморфизм}} & \multicolumn{1}{c|}{\textbf{Генотип}}
		\\ \hline 
		\endhead
		
		\hiderowcolors
		\hline \multicolumn{4}{|r|}{{продолжение на следующей странице...}} \\ \hline
		\endfoot
		
		\hline \hline
		\endlastfoot
		
		\showrowcolors
		
		\BLOCK{ for item in scandat.iterrows(): }
		\VAR{item[1]['Gen']} & \VAR{item[1]['RS']} & \VAR{item[1]['Полиморфизм']} & \VAR{item[1]['Result']} \\
		\BLOCK{ endfor }
		
	\end{longtable}
\end{center}

\vspace{-10mm}
\noindent
\parbox[b][3cm][t]{10mm}{
	\includegraphics[height=25mm]{exclamation.png}}
\hfill
\parbox[b][3cm][t]{100mm}{
	Выявление определенных генетических вариантов не является установлением или подтверждением диагноза; не может служить для диагностики различных зависимостей, а является лишь вспомогательным тестом для врача, позволяющим выбрать оптимальный способ терапии и профилактики.}

\vspace{15mm}
\begin{center}
	\textbf{\BLOCK{ if sex == 'мужской'}
		Уважаемый
		\BLOCK{ else}
		Уважаемая 
		\BLOCK{ endif}
		\VAR{name.split(None, 1)[1]}!}
\end{center}

Перед Вами результат анализа Ваших генов, которые влияют на риск гипертонической болезни и индивидуальную эффективность и безопасность лекарственных средств. Данный отчет может быть полезен для формирования персональных рекомендаций по образу жизни и фармакотерапии для того, чтобы Вы оставались здоровыми и работоспособными.
Наш организм работает по программе, заложенной в генах. Гены определяют структуру и количество биологических молекул, которые обеспечивают те или иные процессы жизнедеятельности. Нарушение строения определенного гена – мутация — это «программная ошибка», которая может приводить к сбою биохимической реакции, лежащей в основе состояния, влияющего на качество жизни (появление болезни). 
Выбор полиморфных вариантов генов и методов их анализа, используемых в настоящем тесте, основан на результатах современных генетических исследований, представленных в международной базе данных научных публикаций Pubmed. 
Мы гарантируем конфиденциальность Ваших личных данных, полученных в настоящем тесте. Обработка Ваших персональных данных осуществляется в соответствии с действующим законодательством РФ.
Благодарим Вас за то, что воспользовались нашей услугой!

\pagebreak
\begin{center}
	\large{Суммарный отчет по результатам молекулярно-генетического анализа}
\end{center}

\begin{center}
	\newcolumntype{C}{>{\arraybackslash}p{3.5cm}}
	\begin{longtable} { |C||p{2.5cm}|p{1.5cm}|p{2.4cm}|p{1.8cm}| }%{ |C||c|c|c|c| }%{ |C||p{2.5cm}|p{1.5cm}|p{2.4cm}|p{1.8cm}| }
		\hline
		Генетический риск & Пониженный \cellcolor{blue!20} 
		& Средний \cellcolor{green!40} 
		& Повышенный \cellcolor{yellow!40} 
		& Высокий \cellcolor{red!50}\\
		\hline
		
		\BLOCK{ for theme in panels[0]['themes']: }
			\multicolumn{5}{|c|}{\VAR{theme["name_report"]}} \\
			\hline
			\BLOCK{ for subtheme in theme['subthemes']: }
				\rowcolor{gray!50}
				\multicolumn{5}{|l|}{\VAR{subtheme["name_report"]}}\\
				\hline
				\BLOCK{ for risk in subtheme['risks']: }
					\VAR{risk["name_report"]} 
					&  
					\BLOCK{ if risk["level"] == 'low'}
					V
					\cellcolor{blue!20}
					\BLOCK{ endif}    
					&
					\BLOCK{ if risk["level"] == 'mid'}
					V
					\cellcolor{green!40}
					\BLOCK{ endif} 
					&   
					\BLOCK{ if risk["level"] == 'high'}
					V
					\cellcolor{yellow!40}
					\BLOCK{ endif} 
					& 
					\BLOCK{ if risk["level"] == 'upper'}
					V
					\cellcolor{red!50}
					\BLOCK{ endif}\\
					\hline
				
				\BLOCK{ endfor }
			\BLOCK{ endfor }
		\BLOCK{ endfor }

	\end{longtable}
\end{center}

\noindent
\parbox[b][3cm][t]{10mm}{
	\includegraphics[height=25mm]{exclamation.png}}
\hfill
\parbox[b][3cm][t]{100mm}{
	Описанные признаки (заболевания) относятся к многофакторным состояниям. Выявленные генетические маркеры не являются диагностическими критериями каких-либо заболеваний. Заключение дано на основании проанализированных генетических маркеров. Другие генетические и не генетические факторы могут влиять на риск оцененных признаков.}

\vspace{15mm}

\textbf{Заключение и рекомендации}\\
\BLOCK{ for theme in panels[0]['themes']: }
	\textbf{\VAR{theme["name_report"]}}\\
	\textit{Рекомендации:}
		\BLOCK{ for subtheme in theme['subthemes']: }
			\BLOCK{ for risk in subtheme['risks']: }
				\BLOCK{ if risk["level"] == 'upper' or risk["level"] == 'high'}
					\VAR{risk["short_recommendation"]}\\ 
				\BLOCK{ endif}
			\BLOCK{ endfor } 
		\BLOCK{ endfor  }
\BLOCK{ endfor }

\pagebreak

\BLOCK{ for theme in panels[0]['themes']: }
	\begin{center}
	\textbf{\VAR{theme["name_report"]}}\\
	\end{center}
	\BLOCK{ for subtheme in theme['subthemes']: }
		\begin{center}
		\textit{\VAR{subtheme["name_report"]}}\\
		\end{center}
		\BLOCK{ for risk in subtheme['risks']: }
			\begin{center}
			{\VAR{risk["name_report"]}}\\
			\end{center}
			
			\textbf{Заключение:}\\
			\VAR{risk["inter"]}\\
			\VAR{risk["recommendation"]}\\
			
			\begin{figure}[bh]
				\noindent\centering{
					\includegraphics[width=120mm]{\VAR{risk['fig_name']}}
				}
			\end{figure}
			
			\begin{center}
				\newcolumntype{C}{>{\arraybackslash}p{3.5cm}}
				\begin{longtable} { |c|c|c|C|C| }
					%\caption*{\textsl{Ваш генотип}}
					\hline
					Ген & RS & Генотип & Функция & Интерпретация \\
					\hline
					
					\BLOCK{ for gene in risk['genes']: }
						\VAR{gene["name"]}
						&
						\VAR{gene["rs_position"]}
						&
						\VAR{gene["result"]}
						&
						\VAR{gene["function"]}
						&
						\VAR{gene["inter"]}\\
						\hline
					\BLOCK{ endfor }
				\end{longtable}
			\end{center}
		
			\pagebreak
		\BLOCK{ endfor } 
	\BLOCK{ endfor  }
\BLOCK{ endfor }
\end{document}
